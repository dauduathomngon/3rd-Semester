\chapter{Đại số tuyến tính}

%%%%%%%%%%%%%%
\section{Vector}

\begin{itemize}
    \item Trong đại số tuyến tính, vector là một đối tượng riêng biệt, khác với các vector trong hình học (hay đúng hơn, vector là phần tử của không gian vector).

    \begin{note}
        Nhưng người ta sẽ không bỏ hoàn toàn các tính chất hình học mà cố gắng đưa các tính chất hình học vào đại số tuyến tính.
    \end{note}

    \item Vector sẽ được hiểu ở dạng vector cột:
    $$
    v = \begin{pmatrix}
        v_1 \\
        v_2 \\
        ... \\
        v_n
    \end{pmatrix}
    $$

    \item Đại lượng vô hướng có thể được hiểu là số thực hoặc số phức (hoặc kĩ hơn thì có thể đọc \href{https://vi.wikipedia.org/wiki/Scalar_(to%C3%A1n_h%E1%BB%8Dc)}{đây})
    
    \item Một vector gồm $n$ phần tử sẽ được gọi là $n$-vector và vector không được kí hiệu là $0_{v}$. Ngoài ra, các phép cộng, trừ, nhân 2 ma trận (hay 2 vector) sẽ không được định nghĩa lại.
\end{itemize}

\begin{defivn}
    Cho $a_1, a_2, ..., a_m$ là các đại lượng vô hướng và $v_1, v_2, ..., v_m$ là các $n$-vector, khi đó:
    $$
    a_1v_1 + a_2v_2 + ... + a_mv_m
    $$
    được gọi là một \textbf{tổ hợp tuyến tính} của các vector $v_1, v_2, ..., v_m$. Các đại lượng vô hướng $a_1, a_2, ..., a_m$ được gọi là \textbf{hệ số} của tổ hợp tuyến tính.
\end{defivn}

\begin{egvn} 
    Trong việc xử lý âm thanh, $v_1, v_2, ..., v_m$ là các vector đại diện cho các tín hiệu âm thanh, được gọi là \textit{audio track}. Tổ hợp tuyến tính $a_1v_1 + a_2v_2 + ... + a_mv_m$ được gọi là \textit{mix} của các audio track, với độ to của audio track được cho bởi $|a_1|, |a_2|, ..., |a_m|$.
\end{egvn}

\begin{defivn}
    \textbf{Tích trong} (hay còn gọi là \textit{tích vô hướng}) của 2 $n$-vector $a$ và $b$, kí hiệu là $\langle a, b \rangle$, được định nghĩa như sau:
    $$
    \langle a, b \rangle = a^Tb = a_1b_1 + a_2b_2 + ... + a_nb_n
    $$
    và kết quả của tích trong là \textit{một đại lượng vô hướng}.
\end{defivn}

\begin{itemize}
    \item[] Ta có thể thấy, tích trong của hai n-vector $a$ và $b$ sẽ có những tính chất sau đây:

    \item \textbf{Tính giao hoán}: $\langle a, b \rangle = \langle b, a \rangle \implies a^Tb = b^Ta$.
    \item \textbf{Tính kết hợp với phép nhân đại lượng vô hướng}: $\langle \beta a, b \rangle = \beta \langle a, b \rangle \implies (\beta a)^Tb = \beta a^Tb$ với $\beta$ là một đại lượng vô hướng bất kì.
    \item \textbf{Tính phân phối với phép cộng vector}: $\langle a + b, c \rangle = \langle a, c \rangle + \langle b, c \rangle \implies (a+b)^Tc = a^Tc + b^Tc$.
\end{itemize}

%%%%%%%%%%%%%%%%%%%%%%%%%%%
% \section{Ánh xạ tuyến tính}

% \begin{itemize}
%     \item Kí hiệu $f: \mathbb{R}^n \to \mathbb{R}$ nói rằng ``$f$ là một hàm số, ánh xạ $n$-vector (mỗi phần tử của vector là số thực) sang số thực''.

%     \item Nếu $x$ là $n$-vector, thì $f(x)$ là một đại lượng vô hướng, kí hiệu cho \textbf{giá trị} của hàm số $f$ tại $x$.

%     \item Ta có thể viết:
%     $$
%     f(x) = f(x_1, x_2, ..., x_n)
%     $$
% \end{itemize}

% \begin{defivn}
%     Cho $a$ là một $n$-vector. Khi đó ta định nghĩa hàm số $f$ như sau:
%     $$
%     f(x) = a^Tx = a_1x_1 + a_2x_2 + ... + a_nx_n
%     $$
% \end{defivn}

%%%%%%%%%%%%%%%%%%%%%%%%%%%
\section{Độ lớn và khoảng cách}

\begin{defivn}
    \textbf{Chuẩn} (hay \textit{chuẩn euclid}) của một $n-$vector $x$, được kí hiệu là $||x||$, được định nghĩa như sau:
    $$
    ||x|| = \sqrt{x_1^2 + x_2^2 + ... + x_n^2}
    $$
\end{defivn}

\begin{itemize}
    \item Đôi khi ta còn gọi chuẩn là \textit{độ lớn}.

    \item Dựa vào định nghĩa tích trong phía trên, ta có thể thấy:
    $$
    ||x|| = \sqrt{x_1x_1 + x_2x_2 + ... + x_nx_n} = \sqrt{\langle x, x \rangle}
    $$

    \item Ngoài ra, chuẩn của vector còn có các tính chất sau đây:
    \begin{itemize}
        \item[(1)] $|| \beta x|| = |\beta| ||x||$ với $\beta$ là một đại lượng vô hướng bất kỳ.
        \item[(2)] $||x + y|| \leq ||x|| + ||y||$ (này còn được gọi là bất đẳng thức tam giác).
        \item[(3)] $||x|| \geq 0$ (độ lớn của vector luôn không âm).
        \item $||x|| = 0$ khi và chỉ khi $x = 0_{v}$.
    \end{itemize}

    \item Xem xét chuẩn của tổng 2 $n-$vector $x$ và $y$, ta có:
    $$
    \begin{aligned}
    ||x+y|| &= \sqrt{\langle x+y, x+y \rangle} \\
    &= \sqrt{(x+y)^T(x+y)} \\
    &= \sqrt{x^Tx + x^Ty + y^Tx + y^Ty} \\
    &= \sqrt{||x||^2 + 2x^Ty + ||y||^2}
    \end{aligned}
    $$
\end{itemize}

\begin{defivn}
    \textbf{Khoảng cách} (hay \textit{khoảng cách euclid}) giữa 2 $n-$vector $a$ và $b$, được kí hiệu là $\textbf{dist}(a,b)$, được định nghĩa như sau:
    $$
    \textbf{dist}(a,b) = ||a - b|| = \sqrt{(a_1 - b_1)^2 + ... + (a_n - b_n)^2}
    $$
\end{defivn}

\begin{itemize}
    \item Ta có định lý Cauchy có thể áp dụng vào đại số tuyến tính:
    $$
    |a^Tb| \leq ||a|| ||b||
    $$

    \item Dùng định lý Cauchy ta chứng minh được tính chất (2) của chuẩn vector:
    $$
    \begin{aligned}
        ||x+y||^2 &= ||x||^2 + 2x^Ty + ||y^2|| \\
        &\leq ||x||^2 + 2||x|| ||y|| + ||y^2|| \\
        &= (||x|| + ||y||)^2 \hspace{10pt} \text{(do chuẩn không âm nên ta có thể bỏ đi bình phương)}
    \end{aligned}
    $$

    \item \textbf{Góc} $\theta$ giữa 2 $n-$vector $a,b$ được định nghĩa như sau:
    $$
    \theta = \arccos \left( \dfrac{a^Tb}{||a|| ||b||} \right)
    $$

    \item Góc giữa $a$ và $b$ được viết là $\angle (a,b)$ với đơn vị là radian.

    \item Ngoài ra ta có:
    \begin{itemize}
        \item $\angle (a,b) = \angle (b,a)$.
        \item $\angle(\alpha a, \beta b) = \angle(a, b)$ với $\alpha, \beta$ là các đại lượng vô hướng bất kì.
    \end{itemize}

    \item Áp dụng góc vào chuẩn tổng của 2 vector $a, b$, ta có:
    $$
    \begin{aligned}
    ||x + y||^2 &= ||x||^2 + 2x^Ty + ||y||^2 \\
    &= ||x||^2 + 2||x||||y|| \cos(\theta) + ||y||^2 \\ 
    &\text{(giống với định lý cosin trong hình học)}.
    \end{aligned}
    $$
\end{itemize}

%%%%%%%%%%%%%%%%
\section{Độc lập tuyến tính}

\begin{defivn}
    Cho  $k$ $n-$vector $v_1, v_2, ..., v_k$ (với $k \geq 1)$ và $a_1, a_2, ..., a_k$ là các đại lượng vô hướng, nếu tổ hợp tuyến tính:
    $$
    a_1v_1 + a_2v_2 + ... + a_kv_k = 0_{v}
    $$
    chỉ có duy nhất một nghiệm $a_1 = a_2 = ... = a_k = 0$, thì $v_1, v_2, ..., v_k$ được nói là \textbf{độc lập tuyến tính}. Ngược lại, tức là có nhiều hơn một nghiệm thì ta nói \textbf{phụ thuộc tuyến tính}.
\end{defivn}
\pagebreak

\begin{itemize}
    \item Ta thấy nếu các vector phụ thuộc tuyến tính với nhau, thì sẽ có ít nhất một vector là tổ hợp tuyến tính của các vector còn lại.
\end{itemize}

\begin{propvn}
    Cho các vector $v_1, v_2, ..., v_n$ độc lập tuyến tính. Xét một vector $x$ bất kì, nếu $x$ là tổ hợp tuyến tính của $v_1, v_2, ..., v_n$ thì $x$ là \textbf{duy nhất}.
\end{propvn}
\begin{proofvn} \vphantom{}
    \begin{itemize}
        \item Nếu $x$ không là duy nhất, khi đó ta có thể biểu diễn $x$ thành 2 tổ hợp tuyến tính của $v_1, v_2, ..., v_n$ (nghĩa là có các bộ hệ số khác nhau).
        
        \item Với bộ hệ số $\beta_1, \beta_2, ..., \beta_n$, ta có:
        $$
        x = \beta_1 v_1 + \beta_2 v_2 + ... + \beta_n v_n
        $$

        \item Với bộ hệ số $\alpha_1, \alpha_2, ..., \alpha_n$, ta có:
        $$
        x = \alpha_1 v_1 + \alpha_2 v_2 + ... + \alpha_n v_n
        $$

        \item Kết hợp cả 2 phương trình trên, ta được:
        $$
        \begin{aligned}
        \alpha_1 v_1 + ... + \alpha_n v_n &= \beta_1 v_1 + \beta_2 v_2 + ... + \beta_n v_n \\
        (\alpha_1 - \beta_1)v_1 + ... + (\alpha_n - \beta_n)v_n &= 0_{v}
        \end{aligned}
        $$

        \item Mà $v_1, v_2, ..., v_n$ độc lập tuyến tính nên $(\alpha_i - \beta_i) = 0 \implies \alpha_i = \beta_i$. Do đó $x$ là duy nhất.
    \end{itemize}
\end{proofvn}

\begin{propvn}
    Một tập hợp gồm các $n-$vector độc lập tuyến tính với nhau sẽ có tối đa $n$ phần tử
\end{propvn}
\begin{proofvn}
    Mệnh đề này là một cách diễn đạt khác của \textbf{Replacement Theorem} (hay \textit{định lý thay thế}), có thể đọc thêm để thấy được cách chứng minh. Diễn đạt lại mệnh đề 2, ta có: ``Một tập hợp gồm $n+1$ hoặc nhiều hơn $n-$vector thì phụ thuộc tuyến tính''.
\end{proofvn}

\begin{defivn} 
\end{defivn}